\documentclass[11pt]{article}

\usepackage{times} % Times New Roman font
\usepackage{graphicx} % For images
\usepackage{enumitem} % For numbered items
\usepackage{lipsum} % For dummy text
\usepackage{hyperref} % For references
\hypersetup{
    colorlinks=true,
    linkcolor=blue,
    filecolor=magenta,      
    urlcolor=cyan,
} % For hyperlinks
\usepackage{amsmath} % For math equations
\usepackage{cleveref} % For clever references
\crefname{appendix}{Appendix}{Appendices}
\usepackage{fancyvrb} % For code display
\usepackage{fancyhdr} % For custom headers
\usepackage{pythonhighlight} % For Python code
\usepackage{float} % For image positioning
\usepackage{amsmath} % For math equations
\usepackage{xcolor} % For custom colors
\usepackage{natbib} % For bibliography
\usepackage{appendix} % For appendices
\usepackage[font={small,it}]{caption} 
% \usepackage{mwe}% For custom captions
\usepackage{listings}
\usepackage{color}
\usepackage{amsmath}
\usepackage{graphicx}
\usepackage{hyperref}
\usepackage{geometry}
\usepackage{caption}
\captionsetup{font={small, sf}}
\usepackage{ctex}
\usepackage{siunitx}
\usepackage{booktabs}
\renewcommand{\tablename}{Table}
\renewcommand{\figurename}{Figure}
\renewcommand{\abstractname}{Abstract}
\geometry{a4paper, margin=1in}
\definecolor{codegreen}{rgb}{0,0.6,0}
\definecolor{codegray}{rgb}{0.5,0.5,0.5}
\definecolor{codepurple}{rgb}{0.58,0,0.82}
\definecolor{backcolour}{rgb}{0.95,0.95,0.92}

\lstdefinestyle{mystyle}{
    backgroundcolor=\color{backcolour},
    commentstyle=\color{codegreen},
    keywordstyle=\color{magenta},
    numberstyle=\tiny\color{codegray},
    stringstyle=\color{codepurple},
    basicstyle=\ttfamily\footnotesize,
    breakatwhitespace=false,
    breaklines=true,
    captionpos=b,
    keepspaces=true,
    numbers=left,
    numbersep=5pt,
    showspaces=false,
    showstringspaces=false,
    showtabs=false,
    tabsize=2
}

\lstset{style=mystyle}

\lstnewenvironment{outputlisting}{
  \lstset{
    basicstyle=\ttfamily\footnotesize,
    frame=single,
    backgroundcolor=\color{lightgray},
    keywordstyle=\color{blue},
    breaklines=true,
    postbreak=\mbox{\textcolor{red}{$\hookrightarrow$}\space},
    xleftmargin=.1\textwidth,
    xrightmargin=.1\textwidth,
    captionpos=b,
    language=[Sharp]C,
    numbers=none,
    tabsize=2,
    showstringspaces=false
  }
}{}

\title{Complex Network Analysis of Classical Chinese Literature: Insights from the Confucian Canon of Scripta Sinica Corpus}
\author{Yichi Zhang u7748799}
\date{May, 2024}

\begin{document}

\maketitle

\thispagestyle{fancy}

\begin{abstract}
    The study of language 
\end{abstract}

\section{Introduction}

\section{Data and Methods}

\subsection{Scripta Sinica Corpus}

\subsection{Network Construction}

\subsubsection{Character Co-occurrence Network (CCN)}

\subsubsection{Character-Sentence Network (CSN)}

\subsection{Network Analysis Methods}

\section{Models and Results}

\subsection{Small-World Properties}

\subsection{Scale-Free Characteristics and Matthew Effect}

\subsection{Hierarchical Structure and Disassortativity}

\section{Discussion}

\section{Conclusion}

\section*{Acknowledgments}

\section*{Ethical Considerations and Broader Impacts}

\section*{References}

\appendix
\section*{Appendices}
\section*{Supplementary Materials}
\subsection*{Network Visualization}

\subsection*{Statistical Tests}


\subsection*{Code and Data Availability}


% References
% \begin{thebibliography}{9}
%   \bibitem{example}
%   Author,
%   \emph{Title}.
%   Publisher, Year.
% \end{thebibliography}
% \bibliographystyle{plain}
% \bibliography{references}  % replace with the actual filename of your .bib file, without the extension

\end{document}
